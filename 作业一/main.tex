%! Tex program = xelatex   
\documentclass{homework}
\usepackage{xeCJK}
\usepackage{amsmath}
\usepackage{booktabs} %表格
% \setmainfont{Times New Roman}
\setCJKmainfont{Kaiti SC}
% \setCJKfamilyfont{song}{Songti SC}
% \renewcommand{\baselinestretch}{1.5} %行间距
\author{朱浩泽 1911530}
\class{智能计算系统作业一}
\date{\today}
\title{\Large{智能计算系统第一次作业}}

\graphicspath{{./media/}}

\begin{document} \maketitle

\question \large{假设只有一个隐层的多层感知机,其输入、隐层、输出层的神经元个数分别为33、512、10,那么这个多层感知机总共有多少个参数是可以被训练的?}\\

\normalsize
答:有$33\times 512+512\times 10 +2 = 22018$个参数是可以被训练的。
\\ 
\\ 
\\ 
\\ 


\question \large{
	$$
	W^{(1)} = 
	\begin{bmatrix}
		w^{(1)}_{1, 1}& w^{(1)}_{1, 2}& w^{(1)}_{1, 3}&\\
		w^{(1)}_{2, 1}& w^{(1)}_{2, 2}& w^{(1)}_{2, 3}&\\
		w^{(1)}_{3, 1}& w^{(1)}_{3, 2}& w^{(1)}_{3, 3}&\\
	\end{bmatrix}
	= 
	\begin{bmatrix}
		0.25& 0.15& 0.30\\
		0.25& 0.20& 0.35\\
		0.10& 0.25& 0.15
	\end{bmatrix}
	$$
	假定输入数据 $x_1 = 0.02, x_2 = 0.04, x_3 = 0.01$\\
	截距 $b_1 = 0.4, b_2 = 0.7$\\
	期望输出 $y_1 = 0.9, y_2 = 0.5$
$$
W^{(1)} = 
\begin{bmatrix}
	w^{(1)}_{1, 1}& w^{(1)}_{1, 2}& w^{(1)}_{1, 3}&\\
	w^{(1)}_{2, 1}& w^{(1)}_{2, 2}& w^{(1)}_{2, 3}&\\
	w^{(1)}_{3, 1}& w^{(1)}_{3, 2}& w^{(1)}_{3, 3}&\\
\end{bmatrix}
= 
\begin{bmatrix}
	0.25& 0.15& 0.30\\
	0.25& 0.20& 0.35\\
	0.10& 0.25& 0.15
\end{bmatrix}
W^{(2)} = 
\begin{bmatrix}
	w^{(1)}_{1, 1}& w^{(1)}_{1, 2}\\
	w^{(1)}_{2, 1}& w^{(1)}_{2, 2}\\
	w^{(1)}_{3, 1}& w^{(1)}_{3, 2}\\
\end{bmatrix}
= 
\begin{bmatrix}
	0.40& 0.25\\
	0.35& 0.30\\
	0.01& 0.35
\end{bmatrix}
$$
}\\ 
\normalsize
\\
答:输入到隐藏层计算
\begin{align*}
	v = 
	\begin{bmatrix}
		v_1\\
		v_2\\
		v_3
	\end{bmatrix}
	= w^{(1)^T}x+b_1
	=
	\begin{bmatrix}
		0.25& 0.15& 0.30\\
		0.25& 0.20& 0.35\\
		0.10& 0.25& 0.15
	\end{bmatrix}
	\begin{bmatrix}
		0.02\\
		0.04\\
		0.01
	\end{bmatrix}
	+0.4
	=
	\begin{bmatrix}
		0.416\\
		0.4135\\
		0.4215
	\end{bmatrix}
\end{align*}
\begin{align*}
	h = 
	\begin{bmatrix}
		h_1\\
		h_2\\
		h_3
	\end{bmatrix}
	= 1+ \frac{1}{1 + e^{-v}}
	= 
	\begin{bmatrix}
		1+ \frac{1}{1 + e^{-0.416}}\\
		1+ \frac{1}{1 + e^{-0.4135}}\\
		1+ \frac{1}{1 + e^{-0.4215}}
	\end{bmatrix}
	= 
	\begin{bmatrix}
		0.6025\\
		0.6019\\
		0.6038
	\end{bmatrix}
\end{align*}
隐含层到输出层计算
$$
\begin{bmatrix}
	z_1\\z_2
\end{bmatrix}
=W^{(2)^T}h + b_2 = 
\begin{bmatrix}
	0.40& 0.35& 0.01\\
	0.25& 0.30& 0.35
\end{bmatrix}
\begin{bmatrix}
	0.6025\\ 0.6019\\ 0.6038
\end{bmatrix}
+0.7 = 
\begin{bmatrix}
	1.1577\\ 1.2425
\end{bmatrix}
$$
$$
\hat{y} = 
\begin{bmatrix}
	\hat{y_1}\\ \hat{{y_2}}
\end{bmatrix}
= \frac{1}{1 + e^{-z}}
= 
\begin{bmatrix}
	\frac{1}{1 + e^{-1.1577}} \\
	\frac{1}{1 + e^{-1.2425}}
\end{bmatrix}
=
\begin{bmatrix}
	0.7609\\ 0.7760
\end{bmatrix}
$$
误差
\begin{align*}
	L(w) = &L_1 + L_2 = \frac{1}{2}(y_1 - \hat{y_1})^2 + \frac{1}{2}(y_2-\hat{y_2})^2\\
	= &\frac{1}{2}(0.7609-0.9)^2 +\frac{1}{2}(0.7760 -0.5)^2\\
	=& 0.0478
\end{align*}
误差相对较大,进行反向传播更新\\
记$w_{2,2}$为$w$\\ 
$$
L(w) = L_1+L_2 = \frac{1}{2} (y_1 - \hat{y} _1)^2 + \frac{1}{2}(y_2 - \hat{y}_2)^2
$$
\begin{align*}
	\frac{\partial L(w)}{\partial w} &= \frac{\partial L(w)}{\partial \hat{y}_1}\cdot \frac{\partial \hat{y}_1}{\partial t_1}\cdot 
	\frac{\partial z_1}{\partial h_2} \cdot 
	\frac{\partial h_2}{\partial v_2}\cdot 
	\frac{\partial v_1}{\partial w}
	+ 
	\frac{\partial L(w)}{\partial \hat{y}_2}\cdot 
	\frac{\partial \hat{y_2}}{\partial z_2}\cdot 
	\frac{\partial z_2}{\partial h_2}\cdot 
	\frac{\partial h_2}{\partial v_2}\cdot 
	\frac{\partial v_2}{\partial w}\\ 
	&=0.6019\times 0.2987\times 0.04\times (-0.3191\times 0.1819\times 0.35 + 0.2760 \times 0.7760 \times 0.2240\times 0.3)\\ 
	& = 0.00005307\\ 
	w & = w - \partial \times \frac{\partial L(w)}{\partial w} = 0.2 - 0.00005 \approx 0.20  
\end{align*}
\end{document}